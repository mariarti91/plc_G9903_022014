\section{Нижний уровень}

\subsection{Требования}

При использовании EAP предполагается что нижний уровень обладает следующими свойствами:

\begin{enumerate}
 \item Ненадежное соединение. В EAP аутентификатор продолжает отправку запроса пока не получит ответ, таким образом EAP не считает соединение надежным. Так, EAP определяет собственный механизм повторной отправки, можно (хотя не желательно) использовать механизм повторной отправки на нижнем уровне и уровне EAP одновременно когда EAP запускается поверх надежного соединения на нижнем уровне.

Необходимо помнить что Success и Failure пакеты не отправляются повторно. Без надежного нижнего уровня и с количеством ошибок сети которым нельзя пренебреч, пакеты могут потеряться из-за таймаутов. Поэтому желательно, что бы реализация была устойчива к патере Success и Failure пакетов, как это описанов в пункте 4.2.

 \item Детектирование ошибок нижнего уровня. Так как EAP считает что нижний уровень ненадежен, EAP полагается на обнаружение ошибок нижнего уровня (например CRC, контрольные суммы, MIC и т.д.). EAP метод можент не включать в себя MIC, или MIC может подсчитываться не для всх полей пакет, таких как поле Code, Identifier, Length or Type. В результате этого, без возможности обнаружения ошибок на нижнем уровне, акие ошибки будут проникать на уровень EAP или уровень методов EAP, что приведет кневозможности пройти аутентификацию.

Например EAP TLS (RFC2716) с подсчитыванием MIC только для поля Type-Data, считает ошибки MIC фатальными. Без возможности обнаружить ошибки на нижнем уровне данный метод (и все похожие) не будут являться надежными.

 \item Безопасность нижнего уровня. EAP не требует от нижнего уровня таких функций безопасности как конфиденциальность каждого пакета, аутентификации, целостности и защиты от повторов. Однако там, где эти функции доступны, могут быть использованы EAP методы, поддерживающие выработку динамической ключевой информации (Пункт 7.2.1). Благодаря этому есть возможность связать EAP аутентификацию последующих данных и предотвращение модификации, спуфинга или повторной отправки. Подробнее об этом в пункте 7.1.

 \item Минимальный MTU. 
\end{enumerate}
