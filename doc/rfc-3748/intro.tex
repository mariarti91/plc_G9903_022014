\section{Введение}

\%TODO 

\subsection{Спецификация требований}

В этой документации некоторые слова используется для описания требований к спецификаций. Слова: ДОЛЖЕН, НЕ ДОЛЖЕН, ОБЯЗАТЕЛЬНЫЙ, РЕКОМЕНДУЕТСЯ, МОЖЕТ и НЕОБЯЗАТЕЛЬНО в данном документе понимаются так, как описано в RFC2119.

\subsection{Терминология}

В документе используются следующие термины:

Аутентификатор \\ Конец линии связи начинающий EAP аутентификацию. Термин аутентификатор описан в семействе стандартов IEEE-802.1X и имеет такое же значение в данном документе.

Пир \\ Другой конец линии связи, реагирующий на аутентификатора. В семействе стандартов IEEE-802.1X это устройство так же известно как Супликант.

Супликант \\  В семействе стандартов IEEE-802.1X другой конец линии связи, реагирующий на аутентификатора. В этому документе такое устройство называется ``пир''.

Базовый сервер аутентификации \\ Базовым сервером аутентификации является субъект, обеспечивающий функционировании службы аутентификации. При использовании данного сервера обычно используется EAP метод аутентификации. Данный термин так же используется в семействе стандартов IEEE-802.1X.

AAA \\ Аутентификация, авторизация и учет. AAA протоколы с поддержкой EAP включены в RADIUS(RFC3579) и DIAMETER(DIAM-EAP). В этом документе термины ``AAA сервер'' и ``Базовый сервер аутентификации'' взаимозаменяемы.

Отображаемые сообщения \\ Данный термен понимается как строка символов, понятная человеку. Сообщения ДОЛЖНЫ представляться в формате UTF-8 RFC2279.

Сервер РПА \\ Объект, который завершает EAP аутентификацию с пиром. В случае когда не используется базовый сервер аутентификации EAP сервер является частью Аутентификатора. В случае если аутнетификатор использует метод передачи пароля EAP сервер находится на сервере базовой аутентификации.

Сброс выполнения \\ Это означает что процес сбрасывается при отсутствии действий. Реализация ДОЛЖНА обеспечивать журналирование событий, в том числе содержимое покетов, которые были проигнориованы и ДОЛЖНА записывать события в счетчик статистики.

Успешная аутентификация \\ В контексте документа ``успешная аутентификация'' это обмен EAP сообщениями после которых Аутентификатор решает дать доступ пиру, а пир решает использовать данный доступ. Решения аутентификатора обычно содержат аспекты аутентификации и авторизации; пир может успешно пройти аутентификацию, но не получить доступ из-за ограничений политики безопасности.

Проверка целостности сообщений (MIC) \\ Ключевая хэш-функция используется для проверки аутентичности и целостности данных. Обычно это называют код аутентификации сообщений(MAC) уровне, но спецификации IEEE 802 использует аббревиатура MIC для избежания коллизий с контроллером доступа к среде (MAC)

Криптографическое разделение \\ Два ключа (х и у) ``криптографически распределимы'' если злоумышленник, знает содержимое всех сообщений, которыми обмениваются стороны и, при этом, не может вычислит x из у и наоборот без ``взлома'' каких либо криптографических предположений. В частности это означает что злоумышленнику известны все временные параметры, передаваемые по открытым каналам связи, и значения всех счетчиков, используемых в протоколе. ``Взлом'' криптографических предположений обычно означает обращение односторонних функций или предсказание значений криптографического генератопа псевдослучайных чисел без знания серета его состояния. Иными словами ключи криптографически разделимы если нельзя просто вычислить х по у и у по х. Что бы злоумышленник выполнил данные вычисления ему необходимо проделать работу эквивалентную полному перебору вариантов.

Мастер ключ сессии (MSK) \\ Ключевой материал распространяется между EAP-пиром и сервером с использованием EAP-методов. MSK менее 64 октетов в длину. В данной реализации AAA сервер действует как EAP сервер: распространяет MSK аутентификатороу.

Расширяемый мастер ключ сессии (EMSK) \\ Дополнительный ключевой материал распространяется между EAP-пиром и сервером с использованием EAP-методов. EMSK менее 4 октетов в длину. EMSK не распространяется на аутентификатор или иные стороны информационного обемена. EMSK зарезервирован для будущего использования и ещё не имеет описания.

Результирующая индикация \\ Способ отображения результата вызванного метода. Получаемые и отправляемые индикаторы:

\begin{enumerate}
\item Пир знает о том что проходит аутентификацию сервера, а так же, её результат.
\item Сервер знает о том что проходит аутентификацию пира, а так же, её результат.
\end{enumerate}

В случае если успешной аутентификации достаточно для предоставления доступа пир и аутентификатор так же узнают существует ли другая сторона, готовая предоставить или получить доступ. Это не всегда может быть так. Аутентифицированному пиру может быть отказано в доступе из-за политики безопасности. Поскольку EAP взаимодейтсвие выполняется между сервером и пиром, в нем могут участвовать други объекты (такие как AAA прокси) и это может влиять на результат. Подробнее это описывается в пункте 7.16.

\subsection{Область применения}

EAP был разработан для аутентификации доступа в сетях, где IP протоколе может быть не доступен. Использование EAP для других целей, таких как транспортировка массивов данных НЕ РЕКОМЕНДУЕТСЯ.

Так как EAP не нуждается в IP протоколе, он обеспечивает достаточную надежность только для транспортровки аутентификационных протоколов, не более того.

EAP - жестко регламентированный протокол, который поддерживает отправку только одного пакета в пролет, в результате чего EAP не эффективно передает информационные сообщения, в отличии от таких протоколов как TCP (RFC793) или SCTP (RFC2960).

EAP поддерживает возможность ретрансляции сообщений, которая берет на себя обязанность пересылки данных в определенном порядке, проверка правильного порядка при получении не поддерживается.

Также EAP не поддерживает фрагментацию и восстановление, методы аутентификации EAP генерируют полезную нагрузку, размер которой привышает минимальный рамер EAP MTU необходимый для поддержки фрагментации.

Аутентификационные методы такие как EAP-TLS (RFC2716) обеспечивают поддержку фрагментации и восстановления, методы EAP описанные в данном документи не обеспечивают поддержку этих механизмов. В результате этого, если EAP пакет превышает размер EAP MTU, то метод сталкивается с определенными проблемами.

Аутентификация EAP начинается сервером (аутентификатором), в то время как большинство протоколов аутентификации начинает клиент (пир). В результате этого может появиться необходимость добавить одно или два дополнительных сообщения (не более одного в каждую сторону) в процесс аутентификации.

В случае поддержки аутентификации при помощи сертификатов количество раундов обмена может быть гораздо больше из-за передачи фрагментов сертификата. В общем случае, фрагментирование EAP пакетов влечет за собой столько сеансов связи сколько необходимо для передачи всего сертификата. Например если сертификат имеет размер 14960 октетов, то для его передачи потребуется 10 сеансов связи с отправкой EAP MTU размером 1496 октетов.

В случае если EAP используется в линияз связи, в которых велика вероятность потери пакетов, или соединение между аутентификаторои или пиром нестабильно EAP методы требующие большого количества сеансов связи могут испытывать трудности. В этом случае целесообразно использовать EAP методы с малым количеством сеансов связи.
