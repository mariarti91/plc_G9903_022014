\section{Расширяемый протокол ауентификации (EAP)} 

\subsection{Спецификация требований}

В этой документации некоторые слова используется для описания требований к спецификаций. Слова: ДОЛЖЕН, НЕ ДОЛЖЕН, ОБЯЗАТЕЛЬНЫЙ, РЕКОМЕНДУЕТСЯ, МОЖЕТ и НЕОБЯЗАТЕЛЬНО в данном документе понимаются так, как описано в RFC2119.

\subsection{Терминология}

В документе используются следующие термины:

Аутентификатор \\ Конец линии связи начинающий EAP аутентификацию. Термин аутентификатор описан в семействе стандартов IEEE-802.1X и имеет такое же значение в данном документе.

Пир \\ Другой конец линии связи, реагирующий на аутентификатора. В семействе стандартов IEEE-802.1X это устройство так же известно как Супликант.

Супликант \\  В семействе стандартов IEEE-802.1X другой конец линии связи, реагирующий на аутентификатора. В этому документе такое устройство называется "пир".

Базовый сервер аутентификации \\ Базовым сервером аутентификации является субъект, обеспечивающий функционировании службы аутентификации. При использовании данного сервера обычно используется EAP метод аутентификации. Данный термин так же используется в семействе стандартов IEEE-802.1X.

AAA \\ Аутентификация, авторизация и учет. AAA протоколы с поддержкой EAP включены в RADIUS(RFC3579) и DIAMETER(DIAM-EAP). В этом документе термины "AAA сервер" и "Базовый сервер аутентификации" взаимозаменяемы.

Отображаемые сообщения \\ Данный термен понимается как строка символов, понятная человеку. Сообщения ДОЛЖНЫ представляться в формате UTF-8 RFC2279.

Сервер РПА \\ Объект, который завершает EAP аутентификацию с пиром. В случае когда не используется базовый сервер аутентификации EAP сервер является частью Аутентификатора. В случае если аутнетификатор использует метод передачи пароля EAP сервер находится на сервере базовой аутентификации.

Сброс выполнения \\ Это означает что процес сбрасывается при отсутствии действий. Реализация ДОЛЖНА обеспечивать журналирование событий, в том числе содержимое покетов, которые были проигнориованы и ДОЛЖНА записывать события в счетчик статистики.

Успешная аутентификация \\ В контексте документа "успешная аутентификация" это обмен EAP сообщениями после которых Аутентификатор решает дать доступ пиру, а пир решает использовать данный доступ. Решения аутентификатора обычно содержат аспекты аутентификации и авторизации; пир может успешно пройти аутентификацию, но не получить доступ из-за ограничений политики безопасности.

Проверка целостности сообщений (MIC) \\ Ключевая хэш-функция используется для проверки аутентичности и целостности данных. Обычно это называют код аутентификации сообщений(MAC) уровне, но спецификации IEEE 802 использует аббревиатура MIC для избежания коллизий с контроллером доступа к среде (MAC)

Криптографическое разделение \\ Два ключа (х и у) "криптографически распределимы" если злоумышленник, знает содержимое всех сообщений, которыми обмениваются стороны и, при этом, не может вычислит x из у и наоборот без "взлома" каких либо криптографических предположений. В частности это означает что злоумышленнику известны все временные параметры, передаваемые по открытым каналам связи, и значения всех счетчиков, используемых в протоколе. "Взлом" криптографических предположений обычно означает обращение односторонних функций или предсказание значений криптографического генератопа псевдослучайных чисел без знания серета его состояния. Иными словами ключи криптографически разделимы если нельзя просто вычислить х по у и у по х. Что бы злоумышленник выполнил данные вычисления ему необходимо проделать работу эквивалентную полному перебору вариантов.

Мастер ключ сессии (MSK) \\ Ключевой материал распространяется между EAP-пиром и сервером с использованием EAP-матодов. MSK менее 64 октетов в длину. В данной реализации AAA сервер действует как EAP сервер: распространяет MSK аутентификатороу.

Расширяемый мастер ключ сессии (EMSK) \\ 
