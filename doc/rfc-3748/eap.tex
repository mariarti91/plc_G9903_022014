\section{Расширяемы протокол аутентификации (EAP)}

Процесс аутентификации EAP происходит следующим образом:

\begin{enumerate}
\item Аутентификатор отправляет запрос аутентификации пиру. Запрос содержит поле ``Type'' в котором указывается что запрашивается. Поле ``Type'' может сообщать о запросах: идентификации, MD5-запрос, и т.д. MD5-запрос близок к описанию из протокола CHAP аутентификации (RFC1994). Обычно, аутентификатор отправляет запрос идентификатора для начала процедуры аутентификации; однако такой запрос не обязателен для протокола и может быть пропущен. Например такой запрос не требуется, если идентификатор определяется поротом, к которому подключен пир (выделенная линия, специализированный свич или диал-ап порт), или другим методом (идентификатор станции свящи, MAC-адрес, и т.д.)
\item Пир посылает ответ на корректны запрос. Как и запрос, ответ содержит поле ``Type'' которое показывает тип запроса.
\item Аутентификатор отправляет дополнительный запрос и пир отвечает на него. Такой обмен происходит столько раз, сколько необходимо для процесса. EAP жестко регламантированный протокол, так что новые запросы, не отправляются пока не придет корректный ответ на предыдущщий запрос. Аутентификатор отвечает за повторную отправку запросов, данный механизм описан в пункте 4.1. После отправки некоторого количества повторных запросов Аутентификатор ДОЛЖЕН завершить EAP передачу данных. При этом Аутентификатор НЕ ДОЛЖЕН отправлять пакеты ``Success'' или ``Failure'' при повторной отправке пакета или когда не удается получить ответ от пира.
\item Обмен продолжаеься до тех пор, пока Аутентификатор не сможет аутентифицировать пира (неприемлемый ответ на один или несколько запросов), в этом случае Аутентификатор ДОЛЖЕН передать сообщение ``EAP Failure'' (Код 4). Так же обмен продолжается до успешной аутентификации пира, в этом случае Аутентификатор ДОЛЖЕН передать сообщение ``EAP Success'' (Код 3).
\end{enumerate}

Преимущества:

\begin{itemize}
\item EAP может поддерживать одновременно несколько протоколов аутентификации без предварительного соглашения о том, какой именно будет использоваться;
\item Сервер сетевого доступа (NAS) (например свич или точка доступа) не знают какой метод аутентификации используется и МОЖЕТ являться прозрачным шлюзом для базового сервера аутентификации. Поддержка сквозного режима является необязательной. Аутентификатор МОЖЕТ аутентифицировать локальные пиры, в то же время выступая шлюзом для не локальных пиров и методов аутентификации реализуемых не локально.
\item Отделение аутентификатора от базового сервера аутентификации упрощает управление полономочиями и политикой безопасности.
\end{itemize}

Недостатки:

\begin{itemize}
\item При использовании PPP, EAP необходимо добавить новый тип аутентификации - PPP LCP и изменить реализацию PPP для использования этого протокола. Это так же откланяет аутентификацию по PPP от спецификации LCP. Так же точки доступа и свичи должны поддерживать реализацию IEEE-802.1X для использования EAP.
\item Если аутентификатор отделяется от базового сервера аутентификации, то усложняется анализ механизмов безопасности и появляется проблема распределения ключей.
\end{itemize}

\subsection{Поддержка последовательностей}


