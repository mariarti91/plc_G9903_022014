\section{Расширяемы протокол аутентификации (EAP)}

Процесс аутентификации EAP происходит следующим образом:

\begin{enumerate}
\item Аутентификатор отправляет запрос аутентификации пиру. Запрос содержит поле ``Type'' в котором указывается что запрашивается. Поле ``Type'' может сообщать о запросах: идентификации, MD5-запрос, и т.д. MD5-запрос близок к описанию из протокола CHAP аутентификации (RFC1994). Обычно, аутентификатор отправляет запрос идентификатора для начала процедуры аутентификации; однако такой запрос не обязателен для протокола и может быть пропущен. Например такой запрос не требуется, если идентификатор определяется поротом, к которому подключен пир (выделенная линия, специализированный свич или диал-ап порт), или другим методом (идентификатор станции свящи, MAC-адрес, и т.д.)
\item Пир посылает ответ на корректны запрос. Как и запрос, ответ содержит поле ``Type'' которое показывает тип запроса.
\item Аутентификатор отправляет дополнительный запрос и пир отвечает на него. Такой обмен происходит столько раз, сколько необходимо для процесса. EAP жестко регламантированный протокол, так что новые запросы, не отправляются пока не придет корректный ответ на предыдущщий запрос. Аутентификатор отвечает за повторную отправку запросов, данный механизм описан в пункте 4.1. После отправки некоторого количества повторных запросов Аутентификатор ДОЛЖЕН завершить EAP передачу данных. При этом Аутентификатор НЕ ДОЛЖЕН отправлять пакеты ``Success'' или ``Failure'' при повторной отправке пакета или когда не удается получить ответ от пира.
\item Обмен продолжаеься до тех пор, пока Аутентификатор не сможет аутентифицировать пира (неприемлемый ответ на один или несколько запросов), в этом случае Аутентификатор ДОЛЖЕН передать сообщение ``EAP Failure'' (Код 4). Так же обмен продолжается до успешной аутентификации пира, в этом случае Аутентификатор ДОЛЖЕН передать сообщение ``EAP Success'' (Код 3).
\end{enumerate}

Преимущества:

\begin{itemize}
\item EAP может поддерживать одновременно несколько протоколов аутентификации без предварительного соглашения о том, какой именно будет использоваться;
\item Сервер сетевого доступа (NAS) (например свич или точка доступа) не знают какой метод аутентификации используется и МОЖЕТ являться прозрачным шлюзом для базового сервера аутентификации. Поддержка сквозного режима является необязательной. Аутентификатор МОЖЕТ аутентифицировать локальные пиры, в то же время выступая шлюзом для не локальных пиров и методов аутентификации реализуемых не локально.
\item Отделение аутентификатора от базового сервера аутентификации упрощает управление полономочиями и политикой безопасности.
\end{itemize}

Недостатки:

\begin{itemize}
\item При использовании PPP, EAP необходимо добавить новый тип аутентификации - PPP LCP и изменить реализацию PPP для использования этого протокола. Это так же откланяет аутентификацию по PPP от спецификации LCP. Так же точки доступа и свичи должны поддерживать реализацию IEEE-802.1X для использования EAP.
\item Если аутентификатор отделяется от базового сервера аутентификации, то усложняется анализ механизмов безопасности и появляется проблема распределения ключей.
\end{itemize}

\subsection{Поддержка последовательностей}

В рамках EAP МОЖЕТ использоваться последовательность методов. Типичным премером этого может быть запрос идентификатора при использовании MD5-вызова. Тем не менее Аутентификатор и пир должны использовать только один метод аутентификации (Type 4 или выше) во время процедуры аутентификации в конце которой Аутнетификатор ДОЛЖЕН отправить ``Success'' или ``Failure''.

После того как пир отправил ответ того же типа что и предыдущий запрос, Аутентификатор НЕ ДОЛЖЕН отправлять другие запросы до окончания процедуры аутентификации (кроме запросов уведомления) и НЕ ДОЛЖЕН отправлять запросы на дополнительные методы аутентификации после окончания выолнения первоначального метода аутентификации; пир не должен реагировать на такие запросы и отбрасывать их как невалидные. Таким образом перезапросы аутентификации не поддерживаются.

Пир не должен отправлять Nak (Наследования или Расширения) в ответ на запрос после того как отправил non-Nak ответ на запрос. Такие поддельные EAP покеты могут быть отправлены атакующим. Аутентификатор при получении неожиданного Nak ответа ДОЛЖЕН его проигнорировать и сделать запись об этом в журнал событий.

Множественные методы аутентификации EAP в рамках одного сеанса не используются так как такие методы уязвимы к атакам ``человек посередине'' (Пункт 7.4) и не совместимы с существующими реализациями.

Если используется метод аутентификации, но внутри него используются другие методы (например тунелирование), то запрет на множественные методы аутентификации не применяется. Такие методы можно рассматривать как один метод аутентификации. Если пир не поддерживает тунелирование, то совместимость может быть обеспечена при помощи начального Nak-запроса (Наследования или Расширения). Для обеспечения должного уровня безопасности методы тунелирования должны иметь механизмы защиты от атака типа ``человек посередине''.

\subsection{Модель мультиплексирования EAP}

Концептуально, реализация EAP состоит из следующих компонентов:

\begin{itemize}
\item Нижний уровень. Нижний уровень отвечает за передачу и прием кадров EAP между пиром и Аутентификатором. EAP запускается поверх различных уровней, таких как PPP, проводные сети 802 (IEEE-802-1X), беспроводные (IEEE 802.11), UDP (L2TP (RFC2661) и IKEv2 (IKEv2)) и TCP (PIC). Работа нижнего уровня описывается в пункте 3.
\item Уровень EAP. Уровень EAP передает и принимает EAP пакеты при помощи нижнего уровня, осуществляет обноружения дублирования и повтор отправки, а так же передает и принимает EAP сообщения с уровня ``Пир и Аутентификатор EAP''
\item Уровень пира и Аутентификатора EAP. 
\item
\end{itemize}
