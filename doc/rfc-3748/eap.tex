\section{Расширяемый протокол ауентификации (РПА)} %РПА = EAP

\subsection{Спецификация требований}

В этой документации некоторые слова используется для описания требований к спецификаций. Слова: ДОЛЖЕН, НЕ ДОЛЖЕН, ОБЯЗАТЕЛЬНЫЙ, РЕКОМЕНДУЕТСЯ, МОЖЕТ и НЕОБЯЗАТЕЛЬНО в данном документе понимаются так, как описано в RFC2119.

\subsection{Терминология}

В документе используются следующие термины:

Аутентификатор \\ Конец линии связи начинающий РПА аутентификацию. Термин аутентификатор описан в семействе стандартов IEEE-802.1X и имеет такое же значение в данном документе.

Пир \\ Другой конец линии связи, реагирующий на аутентификатора. В семействе стандартов IEEE-802.1X это устройство так же известно как Супликант.

Супликант \\  В семействе стандартов IEEE-802.1X другой конец линии связи, реагирующий на аутентификатора. В этому документе такое устройство называется "пир".

Базовый сервер аутентификации \\ Базовым сервером аутентификации является субъект, обеспечивающий функционировании службы аутентификации. При использовании данного сервера обычно используется EAP метод аутентификации. Данный термин так же используется в семействе стандартов IEEE-802.1X.

ААУ \\ Аутентификация, авторизация и учет. ААУ протоколы с поддержкой РПА включены в RADIUS(RFC3579) и DIAMETER(DIAM-EAP). В этом документе термины "ААУ сервер" и "Базовый сервер аутентификации" взаимозаменяемы. %ААУ = AAA

Отображаемые сообщения \\ Данный термен понимается как строка символов, понятная человеку. Сообщения ДОЛЖНЫ представляться в формате UTF-8 RFC2279.

Сервер РПА \\ Объект, который завершает РПА аутентификацию с пиром. В случае когда не используется базовый сервер аутентификации РПА сервер является частью Аутентификатора. В случае если аутнетификатор использует метод передачи пароля РПА сервер находится на сервере базовой аутентификации.

Silently Discard \\ Это означает что процес сбрасывается при отсутствии действий. Реализация ДОЛЖНА обеспечивать журналирование событий, в том числе содержимое покетов, которые были проигнориованы и ДОЛЖНА записывать события в счетчик статистики.
