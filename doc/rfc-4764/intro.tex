\section{"Введение"}

\subsection{"Цели проектирования EAP-PSK"}

Расширяемый протокол аутентификации предоставляет аутентификационный фреймворк, который поддерживает несколько методов аутентификации.

В данном документе описывается метод, который называется EAP-PSK. Данный протокол использует предварительно распределенные ключи (PSK).

EAP-PSK был разработан France Telecom R\&D в 2003-2004. И опубликован как RFC для общей информации в интернет сообществе и для возможности неависимой реализации данного протокола.

Поскольку PSK часто используется в протоколах безопаснсоти, другие протоколы так же могут ссылаться на PSK или ``PSK'' может встречаться в названии протоколов. Например защищенный доступ Wi-Fi (WPA) использует метод аутентификации называется ``WPA-PSK''. EAP-PSK отличается от таких протоколов и его не следует путать с ними.

Целями проектирования EAP-PSK были:
\begin{itemize}
\item Простота: EAP-PSK должен быть легко реализуем и развертываться без каких-либо уже существующих инфраструктур. Протокол должен быть готов быстро, потому что недавно выпущенные протоколы, такие как IEEE 802.11i использует EAP совместно с PPP. Данная ситуация описывается новой моделью угроз и требует ``современных'' методов EAP;
\item Широкая сфера применения: EAP-PSK должен быть пригоден для аутентификации в любых сетях и, в частности, в сетях беспроводных сетях IEEE 802.11;
\item Безопасность: В EAP-PSK должны использваться проверенные, надежные криптографические методы;
\item Расширяемость: Функционал EAP-PSK должен быть легко расширяем.
\end{itemize}

\subsubsection{Простота}
Для простоты, EAP-PSK использует единственный криптографический примитив - AES-128.

Ограничение на криптографические примитив и, в частности, отказ от использования асиметриченого криптографического протокола распределения ключей Диффи-Хелмана делает EAP-PSK:
\begin{itemize}
\item Легким для понимания и реализации, а так же исключает использование криптографического взаимодействия;
\item Легковесным и подходящим для различных типов устройств, особенно для тех, которые обладают малой вычислительной мощностью и небольшим объемом памяти.
\end{itemize}

Однако, как описано в разделе 8 данного документа, это не позволяет внедрить в EAP-PSK дополнительные функции такие как защита персональных данных, поддержка пароля или совершенная прямая секретность (PFS). Этот выбор был сделан сознательно, как компромис между простотой и безопасностью.

Для простоты, в EAP-PSK так же выбран формат сообщений фиксированной длины без соответствия TLV формату.

\subsubsection{Широкая сфера применения}

EAP-PSK был спроктирован с использованием модели угроз в которой злоумышленник имеет полный доступ к среде передачи данных. Данная модель угроз представлена в разделе 7.1 RFC3748.

\subsubsection{Безопасность}

Поскльку разработка механизма обмена ключами авторизации является задачей заведомо трудной и череватой ошибками, EAP-PSK пытается избежать изобретения любого криптографического механизма. Вместо этого предпринята попытка построить его с использваонием существующих примитивов и протоколов которые были описаны криптографическим сообществом.

\subsubsection{Расширяемость}

EAP-PSK явно предусматривает механизм, позволяющий расширять функции протокола в пределах его защищенного канала (Раздел 3.3 данного документа). Благодаря такому механизму EAP-PSK, при необходимости, будет в состоянии обеспечить более сложные сервисы.

\subsection{Терминология}

Аутентификация, авторизаия и учет (AAA) \\ Смотреть RFC2989.

AES-128 \\ Блочный шифр описанный в Advanced Encryption Standart.

Ключ аутентификации (AK) \\ 16-битный ключ получаемый при помощи PSK. Используется при взаимной аутентификации EAP пира и сервера.

AKEP2 \\ Протокол обмена ключами аутентификации. Смотреть ``Entity Authentication and Key Distribution''.

Базовый сервер аутентификации \\ Объект, который предоставляет Аутентификатору услуги проверки подлинности. При использовании, этот сервер, как правило, выполняет методы EAP для Аутентификатора.

CMAC \\ Код проверки подлинности сообщения основанный на шифровании. Это режим аутентификации взятый из AES, рекомендованный NIST в ``Recommendation for Block Cipher Modes of Operation: The CMAC Mode for Authentication''.

Расширяемый протокол аутентификации (EAP) \\ описан в RFC3748.


