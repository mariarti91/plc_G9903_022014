\section{"Введение"}

\subsection{"Цели проектирования EAP-PSK"}

Расширяемый протокол аутентификации предоставляет аутентификационный фреймворк, который поддерживает несколько методов аутентификации.

В данном документе описывается метод, который называется EAP-PSK. Данный протокол использует предварительно распределенные ключи (PSK).

EAP-PSK был разработан France Telecom R\&D в 2003-2004. И опубликован как RFC для общей информации в интернет сообществе и для возможности неависимой реализации данного протокола.

Поскольку PSK часто используется в протоколах безопаснсоти, другие протоколы так же могут ссылаться на PSK или ``PSK'' может встречаться в названии протоколов. Например защищенный доступ Wi-Fi (WPA) использует метод аутентификации называется ``WPA-PSK''. EAP-PSK отличается от таких протоколов и его не следует путать с ними.

Целями проектирования EAP-PSK были:
\begin{itemize}
\item Простота: EAP-PSK должен быть легко реализуем и развертываться без каких-либо уже существующих инфраструктур. Протокол должен быть готов быстро, потому что недавно выпущенные протоколы, такие как IEEE 802.11i использует EAP совместно с PPP. Данная ситуация описывается новой моделью угроз и требует ``современных'' методов EAP;
\item Широкая сфера применения: EAP-PSK должен быть пригоден для аутентификации в любых сетях и, в частности, в сетях беспроводных сетях IEEE 802.11;
\item Безопасность: В EAP-PSK должны использваться проверенные, надежные криптографические методы;
\item Расширяемость: Функционал EAP-PSK должен быть легко расширяем.
\end{itemize}

\subsubsection{Простота}
Для простоты, EAP-PSK использует единственный криптографический примитив - AES-128.

Ограничение на криптографические примитив и, в частности, отказ от использования асиметриченого криптографического протокола распределения ключей Диффи-Хелмана делает EAP-PSK:
\begin{itemize}
\item Легким для понимания и реализации, а так же исключает использование криптографического взаимодействия;
\item Легковесным и подходящим для различных типов устройств, особенно для тех, которые обладают малой вычислительной мощностью и небольшим объемом памяти.
\end{itemize}

Однако, как описано в разделе 8 данного документа, это не позволяет внедрить в EAP-PSK дополнительные функции такие как защита персональных данных, поддержка пароля или совершенная прямая секретность (PFS). Этот выбор был сделан сознательно, как компромис между простотой и безопасностью.

Для простоты, в EAP-PSK так же выбран формат сообщений фиксированной длины без соответствия TLV формату.

\subsubsection{Широкая сфера применения}

EAP-PSK был спроктирован с использованием модели угроз в которой злоумышленник имеет полный доступ к среде передачи данных. Данная модель угроз представлена в разделе 7.1 RFC3748.

\subsubsection{Безопасность}

Поскльку разработка механизма обмена ключами авторизации является задачей заведомо трудной и череватой ошибками, EAP-PSK пытается избежать изобретения любого криптографического механизма. Вместо этого предпринята попытка построить его с использваонием существующих примитивов и протоколов которые были описаны криптографическим сообществом.

\subsubsection{Расширяемость}

EAP-PSK явно предусматривает механизм, позволяющий расширять функции протокола в пределах его защищенного канала (Раздел 3.3 данного документа). Благодаря такому механизму EAP-PSK, при необходимости, будет в состоянии обеспечить более сложные сервисы.

\subsection{Терминология}

Аутентификация, авторизаия и учет (AAA) \\ Смотреть RFC2989.

AES-128 \\ Блочный шифр описанный в Advanced Encryption Standart.

Ключ аутентификации (AK) \\ 16-битный ключ получаемый при помощи PSK. Используется при взаимной аутентификации EAP пира и сервера.

AKEP2 \\ Протокол обмена ключами аутентификации. Смотреть ``Entity Authentication and Key Distribution''.

Базовый сервер аутентификации \\ Объект, который предоставляет Аутентификатору услуги проверки подлинности. При использовании, этот сервер, как правило, выполняет методы EAP для Аутентификатора.

CMAC \\ Код проверки подлинности сообщения основанный на шифровании. Это режим аутентификации взятый из AES, рекомендованный NIST в ``Recommendation for Block Cipher Modes of Operation: The CMAC Mode for Authentication''.

Расширяемый протокол аутентификации (EAP) \\ описан в RFC3748.

Аутентификатор EAP (или просто Аутентификатор) \\ Участник информационного обмена EAP инициирующий запуск методов аутентификации.

Пир EAP (или просто Пир) \\ Участник информационного обмена EAP отвечающий на запросы аутентификации.

сервер EAP (или просто сервер) \\ Объект, который завершает аутентификацию с Пиром. Если нет базового сервера аутентификации, то таковые объектом является Аутентификатор EAP. Если Аутентификатор EAP работает в проходном режиме, то таким объектом является базовый сервер аутентификации.

EAX \\ Режим блочного шифра используемй для аутентификации.

Расширяемый мастер ключ сесси (EMSK) \\ Дополнительный ключевой материал распределяемый при помощи методов EAP между Пиром и сервером. EMSK зарезервирован для будущего использования, он еще не задокументирован и этот ключ не передается третьей стороне. Для более подробного изучения следует обратиться к  ``Extensible Authentication Protocol (EAP) Key Management Framework''. EAP-PSK генерирует 64-байтовый EMSK.

Вектор инициализации (IV) \\ Некоторая последовательность минимум из 64 байт, пригодная для использования в качестве вектора инициализации, которая передается Пиру и серверу. Поскольку IV является известной величиной в таких методах как EAP-TLS, его нельзя использовать для расчета какого либо параметра, который должен оставаться в секрете. В результате чего его использование считается устаревшим и генерация IV не требуется для методов EAP. Для более подробного изучения следует обратиться к ``Extensible Authentication Protocol (EAP) Key Management Framework''

Ключ для генерации ключей (KDK) \\ 16-байтовый ключ, генерируемый для PSK который Пир и сервер используют для генерации сессионных ключей (а именно TEK, MSK, EMSK).

Код аутентификации сообщения (MAC) \\ Неофициально, MAC используется в целях предоставляения гарантии аутентичности и целостности сообщения. В документах IEEE 802.11i используется такая же аббревиатура имеющая другое значение. Это может привести к путанице.

Мастер ключ сессии (MSK) \\ Ключевой материал распределяемый между Пиром и сервером при помощи методв EAP. В существующей реализации AAA сервер, действуя как EAP сервер, передает MSK Аутентификатору. EAP-PSK генерирует 64-байтовый MSK.

Идентификатор сетевого доступа (NAI) \\ Идентификатор используемый для определения участников обмена.

One Key CBC-MAC 1 (OMAK1) \\ Метод генерации аутентификационных кодов сообщений (MAC). CMAC -- имя под которым NIST станадртизировал OMAC1.

Совершенная прямая секретность (PFS) \\ Гарантия того, что компрометация долгосрочного закрытого ключа не приведет к компрометации текущего сессионного ключа. Другми словами, как только диалог EAP заканчивается и используемые им ключи удаляются, никто не сможет восстановить их даже при условии владения всеми долгосрочными ключами Пира и сервера не соверашая при этом полного перебора вариантов. EAP-PSK не имеет такого свойства.

Предварительно заданный ключ (PSK) \\ Данный термин обозначает ключ симметричной криптографии. Такой ключ генерируется и выдается участникам информационного обмена до того как происходит непосредственное использование протокола. Это простая битовая последовательность определенной длины, каждый бит которой был выбран равномерно случайным способом и не зависит от выбора других бит. Для EAP-PSK такой ключ состоит из 16 байт.

Защищенная индикация результата \\ Для подробного изучения следует обратитсья к разделу 7.16 в RFC3748. Данная функция была введена из-за того, что пакеты EAP-Susccess/Failure являются не защищенными и не имеют гарантии доставки.

Переходный EAP ключ (TEK) \\ Сессионнй ключ, используемый для установления защищенного канала между Пиром и сервером во время выполнения протоколов аутентификации EAP. TEK предназначен для использования в криптографическом обмене между Пиром и сервером для защиты EAP взаимодействия. Этот обмен необходим для установления защищенного канала между Пиром и сервером во время EAP аутентификации и не имеет отношения к криптографическим методам, используемым для защиты данных, передаваемых между Пиром и Аутентификатором. EAP-PSK использует 16-байтовый TEK для установления защищенного канала. При криптографическом обмене используется AES-128 в режиме EAX.

\subsection{Условные обозначения}

Все числа представленные в этом документе рассматриваются в сетевом представлении.

|| означает конкатинацию строк (а не логическое ИЛИ).

MAC(K, String) обозначает MAC от строки String вычисленному при помощи ключа K.

[String] обозначает конкатинацию строки с её MAC, вычисление которого зависит от конеткста использования. Например если для вычисления MAC в данный момент используется ключ K, то [String] = String||MAC(K, String).

** обозначает возведение в степень.

``i'' обозначает беззнаковое битовое представление 16 байт целого числа i в сетевом представлении. Следовательно в такой записи могут представляться только числа в интервале от 0 до 2**128 - 1.

<i> обозначает беззнаковое битовое представление четырех байт целого числа i в сетевом представлении. Следовательно в такой записи могут представляться только числа в интервале от 0 до 2**32 - 1.

Ключевые слова МОЖЕТ, ДОЛЖЕН, РЕКОМЕНДУЕТСЯ, ОБЯЗАТЕЛЬНЫЙ, НЕ МОЖЕТ, НЕ ДОЛЖЕН, НЕ РЕКОМЕНДУЕТСЯ, НЕОБЯЗАТЕЛЬНЫЙ понимаются в соответствии с документом ``Key words for use in RFCs to Indicate Requirement Levels''

\subsection{Связь с другими работами}

Во время написания этого документа только три метода EAP были стандартизованы IETF:

\begin{itemize}
\item MD5-Challenge;
\item OTP;
\item GTC.
\end{itemize}

К сожалению все три метода являются устаревшими с точки зрения безопасности. Это частично описано в RFC3748.

Был предложен ряд новых методов:

\begin{itemize}
\item Одним из примеров является экспериментальный RFC (EAP-TLS) который не был стандартизован;
\item Некоторые методы предложены как самостоятельные субпроекты (как например этот проект);
\item И некоторые вовсе не были документированы (Например Rob EAP).
\end{itemize}

Тем не менее еще нет легко и широко доступных методов PSK EAP. Это плохо, так как эти методы являются основными.

Далее представлен краткий обзор предложений по реализации новых методов PSK EAP.

Среди таких предложений есть методы, которые:

\begin{itemize}
\item раскритикованны с точки зрения безопасности (например LEAP и EAP-MSCHAPv2);
\item требуют дополнительной инфраструктуры (например EAP-SIM, EAP-AKA или методы OTP/токен);
\item не являються методами распределения ключей, но часто принимаются за таковые. Это парольные методы (например EAP-SRP, SPEKE);
\item являються базовыми тунельными методами и, по существу, не являються PSK методами так как они требуют сертификат открытого ключа для сервера и позволяет Пиру аутентифицироваться с использованием как методов EAP, так и других, не EAP, методов. Такие методы описаны в документах ``Protected EAP Protocol (PEAP) Version 2'' и ``EAP Tunneled TLS Authentication Protocol (EAP-TTLS)'';
\item заброшенные, но давшие основу EAP-PSK, а именно EAP-Archie;
\item являются возможной заменой EAP-PSK (например EAP-FAST, EAP-IKEv2 и EAP-TLS). 
\end{itemize}

EAP-PSK отлечается от вышеупомянутыхметодов по следующим пунктам:

\begin{itemize}
\item пока не найдено ни одной атаки на EAP-PSK в рамках его модели угроз;
\item EAP-PSK разрабатывался без привзки к конкретной инфраструктуре. Благодаря этому не появляются ограничения на использования протокола и упрощается развертывание проткола ``с нуля''.
\item EAP-PSK пожелал избежать IPR блокировки;
\item EAP-PSK не имеет никаких зависимостей от других протоколов, за исключением EAP;
\item EAP-PSK был ограничен предложением предустановленного ключа и симметричной криптографией для простоты понимания и реализации, и для избежания птенциально сложных конфигураций и обменов данными.
\end{itemize}
