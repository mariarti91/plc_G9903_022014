\section{"Введение"}

\subsection{"Цели проектирования EAP-PSK"}

Расширяемый протокол аутентификации предоставляет аутентификационный фреймворк, который поддерживает несколько методов аутентификации.

В данном документе описывается метод, который называется EAP-PSK. Данный протокол использует предварительно распределенные ключи (PSK).

EAP-PSK был разработан France Telecom R\&D в 2003-2004. И опубликован как RFC для общей информации в интернет сообществе и для возможности неависимой реализации данного протокола.

Поскольку PSK часто используется в протоколах безопаснсоти, другие протоколы так же могут ссылаться на PSK или ``PSK'' может встречаться в названии протоколов. Например защищенный доступ Wi-Fi (WPA) использует метод аутентификации называется ``WPA-PSK''. EAP-PSK отличается от таких протоколов и его не следует путать с ними.

Целями проектирования EAP-PSK были:
\begin{itemize}
\item Простота: EAP-PSK должен быть легко реализуем и развертываться без каких-либо уже существующих инфраструктур. Протокол должен быть готов быстро, потому что недавно выпущенные протоколы, такие как IEEE 802.11i использует EAP совместно с PPP. Данная ситуация описывается новой моделью угроз и требует ``современных'' методов EAP;
\item Широкая сфера применения: EAP-PSK должен быть пригоден для аутентификации в любых сетях и, в частности, в сетях беспроводных сетях IEEE 802.11;
\item Безопасность: В EAP-PSK должны использваться проверенные, надежные криптографические методы;
\item Расширяемость: Функционал EAP-PSK должен быть легко расширяем.
\end{itemize}

\subsubsection{Простота}
Для простоты, EAP-PSK использует единственный криптографический примитив - AES-128.

Ограничение на криптографические примитив и, в частности, отказ от использования асиметриченого криптографического протокола распределения ключей Диффи-Хелмана делает EAP-PSK:
\begin{itemize}
\item Легким для понимания и реализации, а так же исключает использование криптографического взаимодействия;
\item Легковесным и подходящим для различных типов устройств, особенно для тех, которые обладают малой вычислительной мощностью и небольшим объемом памяти.
\end{itemize}
