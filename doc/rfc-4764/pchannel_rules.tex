\newpage
\section{Правила использования Защищенном канале EAP-PSK}

В данном разделе представлено описание:

\begin{itemize}
\item как реализована защищенная индикация результата;
\item детально описана процедура расширенной аутентификации.
\end{itemize}

\subsection{Защищенная индикация результата}

Флаг R в поле PCHANNEL в сообщениях третьего и четвертого типа используется для индикации результата.

Поскольку этот 2-битовый флаг передается по защищенному каналу, то:

\begin{itemize}
\item зашифровывается, так что только Пир и сервер знают значение этого флага;
\item не может быть модифицирован атакующим, так как производится контроль целостности всего поля, а значит и этого флага;
\item защищено от повторения.
\end{itemize}

Флаг R может принимать следующие значения:
\begin{itemize}
\item 01 - CONT;
\item 10 - DONE\_SUCCESS;
\item 11 - DONE\_FAILURE.
\end{itemize}

Пир и сервер помнят последние значения R которые они отправляли и получали друг от друга. Конъюнкция этих двух значений показывает как завершилась операция (успешно или неудачно) в рамках EAP-PSK обмена.

В том случае, если происходит стандартная аутентификация, должен происходить обмен следующими значениями R:

\begin{itemize}
\item Если сервер отправляет значение DONE\_SUCCESS в третьем сообщении EAP-PSK Пиру, а Пир отправляет значение DONE\_SUCCESS в четвертом сообщении EAP-PSK, то считается что аутентификация завершилась успешно.
\item Если сервер отправляет значение DONE\_FAILURE в третьем сообщении EAP-PSK Пиру, а Пир отправляет значение DONE\_FAILURE в четвертом сообщении EAP-PSK, то считается что аутентификация з
авершилась неудачно.
\end{itemize}

В случае расширенной аутентификации может происходить большее число раундов, для учета этих ситуаций вводится значение CONT.

Правила ипользования каждого значения флага R приведены ниже.

\subsubsection{CONT}

И Пир и Сервер инициализируют флаг R, который они собираются отправить значениеи CONT.

Здесь CONT сокращение от ``Continue'' - продолжать. Оно показывает что EAP-PSK обмен ещё не завершен и необходимо продолжение.

В действительности, хотя Пир и сервер должны успешно аутентифицировать друг друга благодаря MAC\_P и MAC\_S для установления защищенного канала, EAP-PSK обмен может всё ещё не считаться успешным даже после успешной взаимной аутентификации, потому что могут возникнуть проблемы на этапе авторизации. Например клиент, использующий беспроводне соединение, успешно проходит аутентификацию, но для пользования услугами должен поплнить свой счет с кредитной карты и для этого ему предоставлен защищенный канал.

\subsubsection{DONE\_SUCCESS}

DONE\_SUCCESS показывает что сторона, отправившая этот флаг, считает что EAP-PSK обмен успешен и предлагает завершение обмена.

После того как сервер отправил значение DONE\_SUCCESS, он должен продолжать отправлять это значение флага R до окончания обмена.

Пир, до отправки DONE\_SUCCESS, должен сначала получить это значение от сервера.

После того, как пир получает значение DONE\_SUCCESS от сервера, он может:

\begin{itemize}
\item отправить CONT серверу, если на его стороне операция не завершилась успехом. Сервер, который получит значение CONT должен продолжить EAP-PSK обмен (В разделе 8.2 представлены рекомендации по безопасной реализации);
\item отправить DONE\_SUCCESS серверу, если намере успешно завершить EAP-PSK обмен;
\item отправить DONE\_FAILURE серверу, если намере завершить EAP-PSK обменc с отрицательным результатом.
\end{itemize}

\subsubsection{DONE\_FAILURE}

DONE\_FAILURE показывает что сторона, отправившая этот флаг, считает что EAP-PSK обмен неудачным и предлагает завершение обмена, потому что считает что этого уже не изменить.

После того как сервер отправил значение DONE\_FAILURE, он должен продолжать отправлять это значение флага R до окончания обмена.

Если Пир отправляет значение DONE\_FAILURE, то сервер, получивший его, обязан сразу же прекратить этот EAP-PSK обмен без отправления уведомлений о неудаче.

\subsection{Расширенная аутентификация}

Начать расширенную аутентификацию может только сервер.

Во время такого EAP-PSK обмена, может быть запущено только одно расширение (идентифицируемое благодаря значению флага EXT\_Type поля EXT).

Так как расширение запускается внутри защищенного канала, оно конфиденциально, гарантируется целостность всех пакетов, а так же защита от повторов.

Для начала процедуры расширенной аутентификации, сервер устанавливает значение флага E = 1 в поле PCHANNEL и передает полезную нагрузку расширения в поле EXT\_Payload.

Так как EAP-PSK не поддерживает фрагментацию данных, расширение не должно отправлять полезную нагрузку более 960 байт, что бы не превысить минимально допустимый размер MTU для EAP в 1020 байт (RFC3748).

Когда Пир получает сообщение третьего типа с флагом E = 1, он проверяет возможность обработки заданного расширения.

Если Пир не в состоянии обработать расширение, то есть он не признает значение EXT\_Type данного расширения, то Пир формирует сообщение четвертого типа с флагом E = 1, а в поле EXT, флаг EXT\_Type устанавливается значение, полученное от сервера, но EXT\_Payload остается пустым.

В зависимости от того, какое значение установлено флагу R, EAP-PSK может:

\begin{itemize}
\item Закончиться
\begin{itemize}
\item Если политика работы Пира предписывает закончить обмен в случае невозможности обработать полученное расширение, то Пир отправляет сообщение четвертого типа, в котором флаг R принимает значение DONE\_FAILURE.
\item Если в сообщении третьего типа сервер отправил флаг R = DONE\_SUCCESS и политика работы Пира позволяет считать аутентификацию успешной, несмотря на невозможность обработки заданного расширения, то Пир отправляет сообщение четвертого типа с R = DONE\_SUCCESS.
\end{itemize}
\item продолжиться ровно на один раунд; а именно, в случае если сервер отправил значение R = CONT в сообщении третьего типа, и политика безопасности воспринимает это действие как успех, даже если не удается обработать расширение, то Пир отвечает сообщением четвертого типа, в котором выставляет R = CONT. В таком случае сервер должен, в соответствии со своей политикой безопасности, отправить DONE\_SUCCESS или DONE\_FAILURE в пятом сообщении. Если сервер отправил в пятом сообщении DONE\_SUCCESS, то Пир должен отправить DONE\_SUCCESS в шестом сообщении. Все эти сообщения должны передаваться со значением E = 1 и полем EXT, в котором EXT\_Type отображает тип текущего расширения, а EXT\_Payload остается пустым (Такое поведение выбрано для облегчения реализации).
\end{itemize}

Если Пир способен обработать предложенное расширение, то он делает это. В этом случае расширение должно иметь доступ к флагам R отправляемых и получаемых сообщений, а так же иметь право обновлять эти значения. Все сообщения, передаваемые между Пиром и сервером, должны передаваться с флагом E = 1, а, также, полем EXT, в котором EXT\_Type отображает тип текущего расширения и не пустым EXT\_Payload.
