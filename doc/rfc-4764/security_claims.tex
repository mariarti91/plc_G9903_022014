\newpage
\section{Требования к безопасности}

Этот раздел указывает требования к безопасности, необходимые для RFC3748.

\begin{itemize}
\item Механизм. EAP-PSK базируется на симметричной криптографии (AES-128) и использует 16-байтовый PSK.
\item Требования к безопасности. EAP-PSK предоставляет:
\begin{itemize}
\item взаимную аутентификацию (раздел 8.1);
\item гарантию целостности (раздел 8.3);
\item защиту от повторов (раздел 8.4);
\item генерацию ключей (раздел 8.7);
\item защиту от атак по словарю (раздел 8.6);
\item независимость сессий (раздел 8.9).
\end{itemize}
\item Стойкость ключей. EAP-PSK использует 16-байтовые ключи.
\item Описание иерархии ключей. Смотреть раздел 2.1.
\item Наличие уязвимостей. EAP-PSK не предоставляет:
\begin{itemize}
\item защиту идентификационных данных (раздел 8.14);
\item конфиденциальность (раздел 8.16);
\item быстрое восстановление соединения (раздел 8.13);
\item возможность фрагментации данных (раздел 8.11);
\item криптографическую привязку (раздел 8.17);
\item выбор криптографических алгоритмов (раздел 8.15);
\item совершенную прямую секретность (раздел 8.10);
\item согласование ключей: сессионный ключ выбирается Пиром (раздел 8.7);
\item привязку канала (раздел 8.12).
\end{itemize}
\end{itemize}
