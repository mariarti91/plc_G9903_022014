\newpage
\section{Криптографическая составляющая EAP-PSK}

EAP-PSK реализован с использованием единственного криптографического примитива - блочного шифра из семейства AES-128. На вход AES-128 принимает 16-байтный предустановленный ключ и блок открытого текста размером 16 байт. После работы алгоритм возвращает 16 байт шифртекста. Детальное описание шифра AES-128 можно увидеть в документе ``Specification for the Advanced Encryption Standard (AES)'' 

AES-128 был выбран потому что:
\begin{itemize}
\item его стандарты и реализации находятся в публичном доступе;
\item алгоритм тщательно изучился криптографическим сообществом и считается безопасным.
\end{itemize}

При реализации EAP-PSK могут использоваться и другие блочные шифры, протокол не зависит от AES-128. Единственными параметрами AES-128 от которых зависит EAP-PSK являются длина ключа и размер блока текста (16 байт). Однако, дял простоты реализации было принято решение использовать только один алгоритм шифрования и не допускать обмена данными во время работы протокола с целью выбора алгоритма из множества. В том случае, если AES-128 окажется устаревшим с точки зрения инфомрационной безопасности, EAP-PSK так же устареет и его необходимо будет заменить протоколом EAP-PSK' который будет использовать другой блочный алгоритм шифрования. EAP-PSK' не должен быть обратно совместим с EAP-PSK из-за сообрежиний безопасности. Следовательно EAP-PSK' должен использовать другие значения поля TYPE в EAP-Reqest/Response. Значения поля TYPE в EAP-Reqest/Response описанны в RFC3748. Использование новых значений поля TYPE будет препятствовать появлению появлению уязвимости выбора протокола.

EAP-PSK использует криптографию для:
\begin{itemize}
\item Установка ключей, во время которй вырабатывается AK и KDK;
\item Обмен ключами аутентификации для взаимной аутентификации и выработки сессионного ключа;
\item Для защиты канала обмена данными во время процедуры аутентификации.
\end{itemize}

Каждый пункт рассмотрен подробнее в последующих пунктах.

\subsection{Установка ключей}
