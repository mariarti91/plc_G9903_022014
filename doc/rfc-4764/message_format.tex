\newpage
\section{Формат сообщений EAP-PSK}

Для простоты, EAP-PSK использует фиксированный формат собщений. Существует четыре различных типа EAP-PSK сообщений:

\begin{itemize}
\item Первый тип, который передается от сервера к Пиру;
\item Второй тип, который передается от Пира к серверу;
\item Третий тип, который передается от сервера к Пиру;
\item Четвертый тип, который отправляется от Пира к серверу. Так же это тип сообщений, которые Пир собирается отправить серверу в случае расширенной аутентификации. Так же это тип сообщений которые сервер будет отправлять Пиру в случае расширенной аутентификации: есть лишь небольшое отличие, в этом случаее EAP устанавливает в поле Code значение 1, в остальных случаях в это поле устанавливается значение 2.
\end{itemize}

Для ясности, весь EAP пакет, который инкапсулируется в сообщение EAP-PSK (то есть сообщение EAP-PSK плюс заголовок EAP), представлен на рисунках \ref{img:message_type1}, 13, 14 и 18. %TODO

\subsection{EAP-PSK сообщение первого типа}

Первый тип сообщений отправляется сервером к Пиру. Формат данного сообщения представлен на рисунке \ref{img:message_type1}.

\begin{figure}[h!]
\center{\includegraphics[width=0.9\linewidth]{./pictures/message_type1}}
\caption{Сообщение первого типа}
\label{img:message_type1}
\end{figure}
