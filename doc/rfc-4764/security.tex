\newpage
\section{Соображения безопасности}

RFC3748 выделяет ряд атак на EAP, сам EAP не предоставляет надежный механизм обеспечения безопасности.

В данном разделе описывается заявленные свойства безопасности EAP-PSK, а так же уязвимости и рекомендации по безопасности модели угроз RFC3748.

\subsection{Взаимная аутентификация}

EAP-PSK обеспечивает взаимную аутентификацию.

Сервер признает аутентичность Пира, поскольку Пир может подсчитать корректный MAC и Пир признает аутентичность сервера, так как сервер также может подсчитать корректный MAC.

Протокол аутентификация, используемый при создании EAP-PSK, AKEP2 доказал свою надежность и надежность методов которые он использует (``Entity Authentication and Key Distribution'').

Алгоритм подсчета MAC, используемый в реализации AKEP2 в EAP-PSK, CMAK так же является проверенным и надежным алгоритмом (``OMAC: One-Key CBC MAC''). Длина Тэга в 16 байт для CMAK так же признана целесообразной криптографическим сообществом.

Основой всех методов служит блочный шифр AES-128, который так же доказал свою надежность.

В заключенни, ключ используемый для взаимной аутентификации, AK, больше не используется ни в каких операциях, благодря чему он обосабливается от остальной части протокола.

EAP-PSK обеспечивает взаимную аутентификацию при условии что используется пара достаточно надежный PSK. В иных случаях взаимная аутентификация невозможна. В этом отношении EAP-PSK ничем не отличается от остальных протоколов использующих Pre-Shared Keys.
