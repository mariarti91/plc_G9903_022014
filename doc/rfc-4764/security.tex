\newpage
\section{Соображения безопасности}

RFC3748 выделяет ряд атак на EAP, сам EAP не предоставляет надежный механизм обеспечения безопасности.

В данном разделе описывается заявленные свойства безопасности EAP-PSK, а так же уязвимости и рекомендации по безопасности модели угроз RFC3748.

\subsection{Взаимная аутентификация}

EAP-PSK обеспечивает взаимную аутентификацию.

Сервер признает аутентичность Пира, поскольку Пир может подсчитать корректный MAC и Пир признает аутентичность сервера, так как сервер также может подсчитать корректный MAC.

Протокол аутентификация, используемый при создании EAP-PSK, AKEP2 доказал свою надежность и надежность методов которые он использует (``Entity Authentication and Key Distribution'').

Алгоритм подсчета MAC, используемый в реализации AKEP2 в EAP-PSK, CMAK так же является проверенным и надежным алгоритмом (``OMAC: One-Key CBC MAC''). Длина Тэга в 16 байт для CMAK так же признана целесообразной криптографическим сообществом.

Основой всех методов служит блочный шифр AES-128, который так же доказал свою надежность.

В заключенни, ключ используемый для взаимной аутентификации, AK, больше не используется ни в каких операциях, благодря чему он обосабливается от остальной части протокола.

EAP-PSK обеспечивает взаимную аутентификацию при условии что используется пара достаточно надежный PSK. В иных случаях взаимная аутентификация невозможна. В этом отношении EAP-PSK ничем не отличается от остальных протоколов использующих Pre-Shared Keys.

\subsection{Защищенная индикация реузультата}

EAP-PSK предоставляет защищенную индикацию результата благодаря 2-битовому R флагу (Раздел 6.1). Этот флаг R защищен благодаря шифрованию и контролю целостности (Раздел 3.3).

Можно предпринять меры против ``Византийской ошибки'', то есть, например, когда Пир пытается втянуть сервер в бесконечный обмен данными. Это может произойти, например, в случае если Пир продолжеает отправлять значение CONT после получения от сервера значения DONE\_SUCCESS. Политика работы может ограничить количество раундов в EAP-PSK обмене, для того что бы избежать данной угрозы, которая выходит за рамки нашей модели угроз.

Также стоит отметить, что криптографическая защита результирующей индикации не предотвращает удаление сообщений.

Например рассмотрим сценарий, в котором:
\begin{itemize}
\item сервер отправил Пиру DONE\_SUCCESS;
\item Пир ответил серверу DONE\_SUCCESS.
\end{itemize}

В случае если последнее сообщение от Пира будет перехвачено и EAP Success отправлен Пиру без каких либо повторов от сервера или повторы от сервера также удаляются, то Пир уверен в успешном окончании аутентификации на сервере, в то время как сервер будет считать иначе.

Такие случаи хорошо известны (``Knowledge and common knowledge in a distributed environment'') и, в некотором смысле, неизбежны. Существует компромис между эффективностью и качеством информационного обмена. EAP-PSK позволяет произойти только одному раунду обмена DONE\_SUCCESS поскольку считается что:

\begin{itemize}
\item Если есть злоумышленник, способный разрушить канал связи, то он может сделать это в любое время (будь это 1 или 10 раунд, или во время обмена данными);
\item Другие приложения/уровни начинают процедуры обмена ключами и проецдуры подтверждения использую ключи, которые генерирует EAP-PSK. Обычно это делоется при помощи IEEE 802.11i ``four-way handshake''. В случае если ошибка не обнаружена EAP-PSK, она должна быть обнаружена на этом этапе (однако стоит отметить что полагаться на сторонние механизмы для обеспечения синхронизации если эта задача не является явной для этих механизмов, это плохая практика).
\end{itemize}

\subsection{Обеспечение целостности}

EAP-PSK обеспечивает целостность благодаря Тэгу его защищенного канала (раздел 3.3).


