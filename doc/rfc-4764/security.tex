\newpage
\section{Соображения безопасности}

RFC3748 выделяет ряд атак на EAP, сам EAP не предоставляет надежный механизм обеспечения безопасности.

В данном разделе описывается заявленные свойства безопасности EAP-PSK, а так же уязвимости и рекомендации по безопасности модели угроз RFC3748.

\subsection{Взаимная аутентификация}

EAP-PSK обеспечивает взаимную аутентификацию.

Сервер признает аутентичность Пира, поскольку Пир может подсчитать корректный MAC и Пир признает аутентичность сервера, так как сервер также может подсчитать корректный MAC.

Протокол аутентификация, используемый при создании EAP-PSK, AKEP2 доказал свою надежность и надежность методов которые он использует (``Entity Authentication and Key Distribution'').

Алгоритм подсчета MAC, используемый в реализации AKEP2 в EAP-PSK, CMAK так же является проверенным и надежным алгоритмом (``OMAC: One-Key CBC MAC''). Длина Тэга в 16 байт для CMAK так же признана целесообразной криптографическим сообществом.

Основой всех методов служит блочный шифр AES-128, который так же доказал свою надежность.

В заключенни, ключ используемый для взаимной аутентификации, AK, больше не используется ни в каких операциях, благодря чему он обосабливается от остальной части протокола.

EAP-PSK обеспечивает взаимную аутентификацию при условии что используется пара достаточно надежный PSK. В иных случаях взаимная аутентификация невозможна. В этом отношении EAP-PSK ничем не отличается от остальных протоколов использующих Pre-Shared Keys.

\subsection{Защищенная индикация реузультата}

EAP-PSK предоставляет защищенную индикацию результата благодаря 2-битовому R флагу (Раздел 6.1). Этот флаг R защищен благодаря шифрованию и контролю целостности (Раздел 3.3).

Можно предпринять меры против ``Византийской ошибки'', то есть, например, когда Пир пытается втянуть сервер в бесконечный обмен данными. Это может произойти, например, в случае если Пир продолжеает отправлять значение CONT после получения от сервера значения DONE\_SUCCESS. Политика работы может ограничить количество раундов в EAP-PSK обмене, для того что бы избежать данной угрозы, которая выходит за рамки нашей модели угроз.

Также стоит отметить, что криптографическая защита результирующей индикации не предотвращает удаление сообщений.

Например рассмотрим сценарий, в котором:
\begin{itemize}
\item сервер отправил Пиру DONE\_SUCCESS;
\item Пир ответил серверу DONE\_SUCCESS.
\end{itemize}

В случае если последнее сообщение от Пира будет перехвачено и EAP Success отправлен Пиру без каких либо повторов от сервера или повторы от сервера также удаляются, то Пир уверен в успешном окончании аутентификации на сервере, в то время как сервер будет считать иначе.

Такие случаи хорошо известны (``Knowledge and common knowledge in a distributed environment'') и, в некотором смысле, неизбежны. Существует компромис между эффективностью и качеством информационного обмена. EAP-PSK позволяет произойти только одному раунду обмена DONE\_SUCCESS поскольку считается что:

\begin{itemize}
\item Если есть злоумышленник, способный разрушить канал связи, то он может сделать это в любое время (будь это 1 или 10 раунд, или во время обмена данными);
\item Другие приложения/уровни начинают процедуры обмена ключами и проецдуры подтверждения использую ключи, которые генерирует EAP-PSK. Обычно это делоется при помощи IEEE 802.11i ``four-way handshake''. В случае если ошибка не обнаружена EAP-PSK, она должна быть обнаружена на этом этапе (однако стоит отметить что полагаться на сторонние механизмы для обеспечения синхронизации если эта задача не является явной для этих механизмов, это плохая практика).
\end{itemize}

\subsection{Обеспечение целостности}

EAP-PSK обеспечивает целостность благодаря Тэгу его защищенного канала (раздел 3.3).

EAP-PSK обеспечивает целостность при условии что используется надежная пара PSK. В ином случае целостноть не гарантируется.  В этом отношении EAP-PSK ничем не отличается от остальных протоколов использующих Pre-Shared Keys.

\subsection{Защита от повторов}

EAP-PSK предоставляет защиту от повторов при взаимной аутентификации благодаря использованию случайных чисел RAND\_S и RAND\_P. Поскольку RAND\_S состоит из 128 бит, возможно осуществить 2**64 (или примерно 1.84 * 10**19) успешных аутентификаций до того как числа начнут повторяться. Следовательно, EAP-PSK обеспечивает защиту от повторов во время аутентификации до тех пор, пока RAND\_S и RAND\_P выбираются случайным образом. Случайность имеет решающее значение в безопасности.

EAP-PSK предоставляет защиту от повторов  во время взаимодействия по защищенному каналу благодаря одноразовому числу N (раздел 3.3). Это число устанавливается сервером в 0 и монотонно увеличивается на единицу при отправки каждого нового валидного сообщения. Например если Пир получает от сервера сообщение с N = x, то при ответе Пир полагает N = x+1 и ожидает от сервера новое сообщение с N = x+2. Повторный приме сообщения с N = x от сервера обозначет для уровня EAP на стороне Пира, что необходимо повторить передачу сообщения с N = x+1, данная операция должна быть прозрачной для уровня EAP-PSK.

Пир должен проверить что сервер действительно инициализировал N = 0.

\subsection{Атака отражением}

EAP-PSK предоставляет защиту от атаки отражением в случае расширенной аутентификации, так как:

\begin{itemize}
\item предоставляется защита целостности заголовка EAP (который содержит защищенную индикацию результата);
\item существует два отдельных пространства для одноразовых чисел N: сервер принимает сообщения только с нечетным N, а Пир принимает сообщения только с четным N.
\end{itemize}

\subsection{Атака по словарю}

Поскольку EAP-PSK не является парольным протоколом, он не подвержен атаке по словарю.

Более того, PSK используемый EAP-PSK не должен генерироваться на основании пароля. Если бы PSK генерировался на основании пароля, то атаки по словарю были бы актуальны.

Однако, использование 16-байтового PSK приводит к:

\begin{itemize}
\item эргономическому фактору: некоторым людям может показаться неудобным распределение таких ключей вручную;
\item фактору развертывания: некоторые люди предпочитают использовать при развертывании базы учетных данных, которые содержат пароли, а не PSK.
\end{itemize}

Так как некоторые могут не воспользоваться советом и предпочтут генерировать PSK на основании пароля, такой способ представлен в приложении A. Но данный метод не предотвращает возможность использования атаки по словарю, а только усложняет её использование.

\subsection{Порождение ключей}

EAP-PSK поддерживает порождение ключей.

Иерархия ключей описана в разделе 2.1.

Механизмом для порождения ключей является специалный режим блочного шифра - режим счетчика.

Счетчик, используемый в EAP-PSK соответствует условиям, описанным в ``The Security of One-Block-to-Many Modes of Operation'', что доказывает его надежность.

В качестве блочного шифра используется AES-128, который считается безопасным.

Первое порождение ключей преднозначено для генерации AK и KDK из PSK. Этот этап называется этапом установки ключей (раздел 3.1). На этом этапе PSK используется как ключ для режима счетчика блочного шифра. Таким образом AK и KDK являются криптографически различимы и вычислимы только с использованием PSK. Второе порождение ключей необходимо для получение TEK, MSK и EMSK (раздел 3.2). На этом этапе режим счетчика использует KDK в качестве ключа.

Структура протокола явно подразумевает что ни один из ключей, ни AK, ни KDK не передаются третьей стороне и используеются только Пиром и сервером. AK теряет свою эффективность для взаимной аутентификации в тот момент, когда становится известен третьей стороне. По аналогии, TEK, MSK и EMSK теряют свою эффективность если раскрывается KDK.

Следует отметить, что Пир контролирует сессионные ключи, полученные при помощи EAP-PSK. В частности, Пир может сгенерировать случайное число, отправляемое в EAP-PSK, таким, что одни из девяти 16-байтовых блоков (раздел 2.1)  принимет заранее установленное значение.

Число выбирается без предотвращения этого контроля сессионных ключей Пиром, потому что:

\begin{itemize}
\item Предотвращение такой ситуации приведет к усложнению протокола (как правило, кключение одностороннего режима работы AES для генерации ключей).
\item Считается, что Пир не будет пытаться заставить сервер использовать заранее заданные сеансовые ключи. Такая атака выходит за проделы модели угроз и считается малополезной для Пира, знающего PSK.
\end{itemize}

Тем неменее такое поведение не рекомендуется разделом 7.10 RFC3748.

Так как генерация ключей требует некоторого количества криптографических вычислений, рекомендуется генерировать сессионные ключи только когда аутентификация завершится успешно. (то есть когда сервер и Пир убедятся во взаимной аутентичности путем проверки MAC).

Стоит отнестись с большой осторожностью к реализации генераторов ключей, так как при неудачном EAP-PSK обмене ключи не будут доступны.

TEK не должен быть доступен никому, кроме участников EAP-PSK обмена.

\subsection{Зашита от DoS атак}

При проектирование EAP-PSK не стояло цели разработать защиту от DoS атак.

Это, однако, означает что EAP-PSK не предоставляет никаких очевидных мест для проведения такой атаки.

Стоит отметить что сервер должен производить криптографические вычисления и хранить некоторую информацию сли он участвует в EAP-PSK обмене, а именно генерировать и хранить 16-байтовый RAND\_S. Тем не менее это не должно привести к истощению ресурсов, так как данные операции не трудоёмки и не требуют больших объемов памяти.

Следует отметить, что и Пир и сервер должны фиксировать их RAND\_P и RAND\_S для защиты партнера от флуда.

Рекомендуется не допускать повторения EAP в рамках диалога, что бы предотвратить возможную DoS атаку. Действительно, так как EAP уведосления не проверяются на целостность, злоумышленник может легко сэмитировать такие пакеты. Такой злоумышленник может заставить Пира отправлять EAP уведомления и, таки образом, замедлить его аутентификацию или вынудить Пира к неожиданным действиям(например увведомления могут использоваться для того, что бы ``подсказать'' Пиру выполнить ``плохое'' действие).

При реализации EAP-PSK, или Пира, или сервера следует указать число максимально возможных, неправильных криптографических проверок. Например, приведет ли получение неправильного MAC\_P во втором сообщении EAP-PSK к фатальной ошибке или отбрасыванию покета и ожиданию нового, корректного пакета? Существует компромис между возможностью осуществить обмен несколькими предварительно подделаными пакетами и позволением провести полноценную DoS атаку (в случае если первая ошибка фатальна).

Для простоты и устойчевости перед DoS, EAP-PSK не использует никаких сообщений об ошибке. Следовательно ``недействительные'' EAP-PSK сообщения просто отбрасываются. Хотя это приводит к усложнению процедуры тестирования на совместимость и отладку, это позволяет упростить реализацию и не дает никаких возможностей для проведения DoS атак.

\subsection{Автономность сессий}

Благодаря механизму генерации ключей, EAP-PSK предоставляет автономность сессий: пассивные атаки (такие как сбор трафика при EAP обмене) или активные атаки (в том числе компрометирующих MSK или EMSK) не компрометируют последствий для сгенерированных или будующих MSK или EMSK.

Предположение что RAND\_P и RAND\_S случайны, является основопологающим для безопасности EAP-PSK вообще и автономности сессий в частности.

\subsection{Выявление PSK}

EAP-PSK не обладает свойством совершенной прямой безопасности. Раскрытие PSK приводит к раскрытию всех записанных сессий.

Копрометация PSK позволяет злоумышленнику выдавать себя за Пира или сервера: компрометация PSK приводит к ``полной'' компрометации всех последующих сессий.

EAP-PSK не предоставляет защиту от того, что лигитимный Пир передаст PSK третьей стороне. Такая защита может быть обеспечена при использовании соответствующего хранилища для PSK, выбор которого выходит за рамки данного документа. PSK используемый EAP-PSK должен быть выдан только двум сторонам: Пиру и серверу. В частности, не стоит передавать один и тот же PSK группе устройств, сообщающихся с одним сервером.

PSK используемый EAP-PSK должен быть криптографически разделим с PSK, используемые другими протоколами, иначе EAP-PSK может быть скомпрометирован. Правилом хорошего тона является использование различных ключей для разных целей.

\subsection{Фрагментация}

EAP-PSK не поддерживает фрагментацию и последуюющее восстановление данных.

Более того, пакет EAP-PSK должен иметь размер не более 1015 байт, так как:

\begin{itemize}
\item Максимальная длина для NAI идентификатора Пира в EAP-PSK составляет 966 байт (раздел 5.2). На практике этого ограничения быть не должно (подробности в разделе 2.2 ``The Network Access Identifier'').
\item Максимальная длина для EXT\_Payload в EAP-PSK составляет 960 байт (разделы 5.3 и 5.4).
\end{itemize}

В разделе 3.1 RFC3748 указано что нижний уровень, поверх которого используется EAP, должен обладать размером MTU не менее 1020 байт. С учетом того, что заголовок EAP пакета составляет 5 байт, мы получаем ограничение в 1015 байт для EAP-PSK.

Расширения, которым требуется передавать полезную нагрузку размером более 960 байт, должны иметь собственные механизмы для фрагментации и восстановления данных.

\subsection{Привязка к каналу}

EAP-PSK не предоставляет привязку к каналу так как эта возможность до сих пор находится в процессе разработки (``Authenticated Service Information for the Extensible Authentication Protocol (EAP)'').

Тем не менее эту функцию можно легко добавить в качестве расширения для EAP-PSK (раздел 4.2).

\subsection{Быстрое восстановление соединения}

EAP-PSK не предоставляет никаких возможностей по быстрому восстановлению соединения.

В самом деле, как отмечено, например, в ``Authenticated Key Exchange Secure Against Dictionary attacks'', взаимная аутентификация (без счетчиков или временных меток) требует пересылки трех пакетов, а, в данном случае, четырех, так как в EAP на каждый EAP-Request должен быть отправлен EAP-Response.

Поскольку этот минимум уже достигнут в стандарте EAP-PSK, не существует возможность уменьшить количество раундов в EAP-PSK без использования временных меток или счетчиков. Временные метки и счетчики не использвались сознательно, в пользу простоты и безопасности (например для избежания проблем синхронизации).

\subsection{Защита идентификационных данных}

Поскольку было решено ограничеться единственным криптографическим примитивом из симметричной криптографии, а именно, блочным шифром AES-128, не представляется возможным предоставлять ``разумную'' защиту идентификационных данных без ущерба простоте.

Далее приводиться неформальное объяснение того, что подразумевается под защитой идентификационных данных и логическое обоснование требований к этой защите. За дополнительной информацие необходимо ознакомиться с ``SIGMA: the `SIGn-and-MAc' Approach to Authenticated Diffie-Hellman and its Use in the IKE Protocols''.

Защита идентификационных данных в основном означает предотвращение ракрытия идентификационных данных сторон, взаимрдействующих по сети, что противоречит процессу аутентификации. Существует два уровня защиты идентификационных данных: защищающих от пассивных атак и защищающих от активного воздействия.

Как объясняется в ``SIGMA: the `SIGn-and-MAc' Approach to Authenticated Diffie-Hellman and its Use in the IKE Protocols'', ``распространенным примером [защиты идентификационных данных] является случай, когда мобильные устройства пытаются предотвратить сопоставление их (изменения) места положения с (личностью пользователя) самим устройством''.

Если для защиты идентификационных данных используется только симметричная криптография, то такая защита относительно слаба и сводится к управлению псевдонимами. Другими словами, Пир и сервер согласовывают псевдонимы, которые они будут использовать для своей идентификации. Такие псевдонимы требуют периодического изменения, по возможности, это должно происходить по защищенному каналу, что бы злоумышленник не смог узнать их до того, как они будут использованы первый раз.

При использовании управления псевдонимами, существует компромис между учетеом рассинхронизации псевдонимов (благодаря постоянной иентификации) и уязвимостью к активным атакам (например атакам, при которых злоумышленник подделывает сообщения, инициирую рассинхронизацию псевдонимов).

На самом деле, протовол, использующий переменные псевдонимы может создать ``рассинхронизированную'' ситуацию, например такую, когда Пир уверен что его текущий псевдоним ``pseudo1'' в то время как сервер уверен что этот Пир имеет псевдоним ``pseudo2'' (так как сервер отправил запрос на изменение псевдонима ``pseudo1'').

Поскольку механизмы управления псевдонимами усложнили бы протокол, было решено не использовать эти механизмы в EAP-PSK.

Тем не менее, EAP-PSK тривиально обеспечивает защиту от идентификации ``реального'' пользователя в случае ``обнаружения''. Например возьмем пользователя Ивана Петрова, который ходит и подключается к различным точком беспроводной сети, пренадлежащей компании под названиеи ``ЗЛО''. Предположим, что этот пользователь проходит аутентификацию у себя дома (услуги предоставляет коспания ``ДОБРО'') используея EAP методы с идентификатором (например ip@id.id), что позволяет компании ``ЗЛО'' (или злоумышленнику) получить его ``реальный'' идентификатор (то есть Иван Петров). Что позволяет наприме отправить ему целевую рекламу, с целью изменить поставщика услуг.

EAP-PSK может очень просто помещать такой атаке, просто не предлагаю Ивану Петрову NAI, который бы привел к раскрытию его личности. Считается, что если используется NAI, никак не коррелирующий с ``реальным'' идентификатором, то никакая важная информация не разглашается благодаря методу EAP.

В действительности, идентификатор, используемый Пиром, всё равно будет раскрыт для обеспечения маршрутизации AAA. Более того, MAC-адрес сетевой карты Пира может использоваться для отслеживания Пира так же эффективно, как и фиксированный NAI.

Стоит отметить, что сервер периодически раскрывает свой идентификатор, что может позволить выполнить ``разведку''. Это не должно быть проблемой, так как идентификатор сервера не должен оставаться в тайне.  Напротив, пользователи, как правило, хотят знать, с кем они будут взаимодействовать, что бы выбрать нужную для себя сеть.

\subsection{Выбор криптографических методов}

EAP-PSK не позволяет никаких обменов данными с целью выбора криптографических методов для дальнейшего взаимодействия. Следовательно неуязвим для атак, связанных с выбором криптографических методов и не требует защиты от таких атак.

\subsection{Конфиденциальность}

Хотя EAP-PSK обеспечивает конфиденциальность в рамках своего защищенного канала, он не обеспечивает конфиденциальность в соответствии с разделом 7.2.1 RFC3748: ``Метод должен предоставлять защиту идентификационных данных''.

\subsection{Криптографическая привязка}

Поскольку EAP-PSK не предназначен для использования внутри других протоколов, исключается одноранговая аутентификация и не реализуется криптографическая привязка.

\subsection{Реализация EAP-PSK}

Для того, что бы обеспечить реальную безопасность, протокол должен быть не только хорошо продуман и документирован, но и при его реализации необходимо проявлять осторожность.

Например реализация криптографических алгоритмов требуются специальные навыки, так как криптографическое программное обеспечение является уязвимым не только к классическим атакам (например переполнению стека или отсутствию проверок), но и к специализированным атакам (например атакам по сторонним каналам, например временным, которые описаны в ``Timing Attacks on Implementations of Diffie- Hellman, RSA, DSS, and Other Systems''). В частности стоит проявить осторожность, воизбежании таких атак на реализацию EAX. Пожалуйста, ознакомтесь с ``The EAX mode of operation''.

Реализация EAP-PSK должна использовать хороший источник случайности, для генерации случайных чисел, используемых в протоколе. Подробнее о генераторах случайных чисел для приложений обеспечивающих безопасность можно узнать в ``Randomness Requirements for Security''.

Обработку информации, такой как ключевой материал (PSK, AK, KDK и т.д.), следует реализовывать с учетом рекомендаций по безопасности, описаных, например, в ``National Industrial Security Program Operating Manual''.

Описание хранилища для PSK, который используется EAP-PSK выходит за рамки данного документа. В частности, ничто не мешает хранить PSK на устойчевых к взлому устройствах, таких как смарткарты. Это предпочтительнее, чем запоминать ключ или записывать его на бумаге. Выбор хранилища для PSK может оказывать важное влияние на безопасность.
