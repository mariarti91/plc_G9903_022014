\section{Безопасность}
\subsection{Контроль доступа и аутентификация}

Конечное устройство (КУ) не может получить доступ к сети без предварительной идентификации (сравнением с белым или черным списком) и аутентификации. Идентификация и аутентификация строятся на двух параметрах, которые однозначно определяют КУ:

\begin{itemize}
 \item адрес EUI-48 MAC, описанный в IEEE 802-2001. Данный адрес легко перевести в адрес EUI-64 (описан в IEEE 802.15.4 и связанных с ним документах);
 \item 128-битный общий ключ (предварительный ключ или PSK) используемый как удостоверение в процессе аутентификации. Данный ключ находитяс на КУ и сервере. Аутентификация базируется на проверке PSK сервером и подтверждением для КУ что сервер знает PSK, тоесть происходит взаимная аутентификация. Сам PSK держется в секрете.
\end{itemize}

Процес идентификации и аутентификации запускается при перезапуске КУ и, также, может быть запущен в любой момент сгласно политике безопасности. Все данные для идентификации и аутентификации передаются при помощи самонастраевоемого протокола 6LoWPAN(LBP) (пункт 9.4.4) который внедряет собственный расширяемый протокол аутентификации(EAP) (пункт 9.4.4.2.1.2).

Как показано на рисунке 10-1, LBP и EAP были разработаны так, что бы передаваться по промежуточным узлам. Так, в начале фазы самонастройки, если КУ ещё не получило собственный 16-битный адресс на расстоянии 1 прыжка от PAN координатора(LBS), они всё равно могут общаться напрямую. В иных случаях они должны использовать промежуточный узел(LBA), расположенном на растоянии одного прыжка от LBD.

Кроме того, должны быть продуманы две другие архитектуры механизма аутентификации:
\begin{itemize}
 \item Функция сервера аутентификации непосредственно поддерживается LBS и тогда все материалы для аутентификации(списки доступа, полномочия и т.д.) должны быть загружены в LBS;
 \item Функция сервера аутентификации поддеривается удаленным(централизованно) AAA сервером, в данном случае LBS только отвечает за пересылку EAP сообщений на AAA сервер по стандартному AAA протоколу (то есть RADIUS [IETF RFC 2865]).
\end{itemixe}


